\documentclass[12pt,a4paper]{article}

\usepackage{color}

\usepackage[usenames,dvipsnames,svgnames,table]{xcolor}

%\pagecolor{black}

\newcommand{\wwwrect}{\colorbox{blue}{\color{white}{\textsf{\textbf{\large www}}}}}

\newcommand{\myrect}[1]{\colorbox{blue}{\color{white}{\textsf{\textbf{\large #1}}}}}

\definecolor{narenji}{rgb}{1,0.5,0}

\begin{document}

% Command: \textcolor{colorname}{text}
This part of text has \textcolor{red}{RED} color.\\

% switch: {\color{colorname} text}
This part of text has {\color{blue} BLUE} color.\\
%{\color{white} This part of text has {\color{blue} BLUE} color.}\\

% \colorbox{colorname}{text}
This part of text is in \colorbox{narenji}{Orange Box}.\\

% \fcolorbox{framecolor}{bkgcolor}{text}
This part of text is in \fcolorbox{red}{yellow}{Yellow Box with Red Frame}.\\

% WWW example from Bishop's Book
Such exercises are denoted by \wwwrect.\\

We defined a new macro, with name \verb"wwwrect", which displays the \wwwrect.

And finally, we defined another macro, with name \verb"myrect", which displays Text as \myrect{Text}.


\end{document}
